\documentclass{jarticle}
\usepackage{amsmath}
\usepackage{listings}
\lstset{
  basicstyle = {\ttfamily},
  frame = {tb}
}

\title{計算機システム演習 第7回レポート}
\author{17B13541 \and 細木隆豊}
\date{}

\begin{document}
  \maketitle
  \section{テスト結果}
  \begin{lstlisting}
    first data of rdata1: 0
    first data of rdata2: 0
    second data of rdata1: 100
    second data of rdata2: 0
    third data of rdata1: 100
    third data of rdata2: 0
    fourth data of rdata1: 200
    fourth data of rdata2: 0
    fifth data of rdata1: 200
    fifth data of rdata2: 200
    InstMemory[0x04000000] = 350
    InstMemory[0x04000004] = 100
    InstMemory[0x04000008] = 200
    rdata[0x10000004] : 0
    rdata[0x10000008] : 0
    rdata[0x10000004] : 0
    rdata[0x10000004] : 300
    rdata[0x10000008] : 200
    rdata[0x10000008] : 200
  \end{lstlisting}
  \section{工夫点}
  registerはregister\_write, read1, read2, write1, wdataに、inst\_memoryはaddr, instに、data\_memoryはmem\_write, mem\_read, addr, wdataの値によって変わる箇所がわかりやすいようなtestを実装した。
  \section{感想}
  inst\_memory\_runの一行目の説明"instにデータを送る"がどういう意味なのかわからず、無視してコードを書いたのですが問題ないのでしょうか。
\end{document}
